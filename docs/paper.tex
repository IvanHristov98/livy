\documentclass[a4paper,12pt]{article}

\usepackage[margin=90pt]{geometry}
\usepackage[utf8]{inputenc}
\usepackage[bulgarian]{babel}
\usepackage[unicode]{hyperref}
\usepackage{amsfonts}
\usepackage{amssymb}
\usepackage{enumitem, hyperref}
\usepackage{upgreek}
\usepackage{graphicx}
\usepackage{mathtools}
\usepackage{indentfirst}
\usepackage{csquotes}
\usepackage{mathptmx}

\graphicspath{ {./img/} }

\begin{document}

\maketitle
\begin{center}\textbf{Преддипломна работа}\end{center}

\begin{center}на тема\end{center}

\begin{center}\textbf{Дедупликация на изображения от архиви на културни институци}\end{center}

\pagebreak

\section{Увод}

\subsection{Актуалност на проблема и мотивация}

С течение на времето се правят находки с културна стойност. Осезаема част от тях са под формата на снимки, съдържащи исторически лица редом с характерните за тяхната среда обекти. Анализът на такива снимки изисква участието на историци и етнографи. Следователно процесът е трудоемък и с възможност за допускане на субективни грешки. Дублициране на исторически изображения е възможно, било то поради невнимание при сканиране или сливане на архиви с общи снимки. Повтарянето на изображения е възможно да доведе до излишен човешки труд и съответно загубено време.

\bigbreak

Пример за институция с този пробем е Троянският музей. Той разполага с архив от \textit{4TB} снимки в \textit{JPG} формат, чиито източници варират. Някой от тях са заснети чрез фотоапарат, докато други са сканирани. Съответно те са подложени на различни афинни трансформации (ротация, скалиране, отместване и др.), дисторция и илюминация. Пример за това е, че размерите на изображенията варират от около \textit{200KB} до около \textit{10MB}. Троянският музей специализира в народните занаяти, но има и богата сбирка от градски изображения - някои от които са с тълпи от хора, други с малко на брой субекти. Съответно детайлността на изображенията варира, което би могло да бъде проблем при наличие на ниска резолюция на изображенията. Снимките от музея са исторически, съответно има наличие на черно-бели и цветни изображения.

\bigbreak

Архивът на Троянският музей със своето разнообразие е достатъчно представителен за проблема, но той далеч не е единственият такъв. Пример за подобно множество от снимки е това на Народната библиотека в Пловдив. Следователно дедупликацията на исторически изображения е проблем, чиято автоматизация би имала културен и исторически принос на много места.

\subsection{Цел и задачи на дипломната работа}
\subsection{Очаквани ползи от реализацията}
\subsection{Структура на преддипломната работа}

\section{Преглед на предметната област}

\subsection{Основни дефиниции}
\subsection{Подходи за решаване на проблемите}
\subsection{Съществуващи решения}
\subsection{Избор на критерии за сравнение и сравнителен анализ на решения}
\subsection{Изводи}

\section{Използвани технологии, платформи и/или методологии}

\section{Анализ}

\section{Проектиране}

\section{Реализация, тестване/експерименти и внедряване}

\section{Заключение}

\section{Използвана литература}


\end{document}